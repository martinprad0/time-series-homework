% chktex-file 3 chktex-file 9 chktex-file 12 chktex-file 17 chktex-file 18 chktex-file 36 chktex-file 40
\section*{Exercise 4.9}

Let \(\{X_t\}\) be the process
\[
X_t = A\cos(\pi t / 3) + B\sin(\pi t / 3) + Y_t
\]
where \(Y_t = Z_t + 2.5Z_{t-1}, \{Z_t\} \sim \mathrm{WN}(0, \sigma^2)\), and \(A\) and \(B\) are uncorrelated 
\(\mathcal{N}(0, v^2)\) random variables which are also uncorrelated with \(\{Z_t\}\). Find the covariance 
function and the spectral distribution function of \(\{X_t\}\).

\subsection*{Solution Autocovariance}

For \textbf{Exercise 1.7(c)} I proved that for $Z_1, Z_2 \sim \mathcal{N}(0,\sigma^2)$, the process $Z_1 \cos(ct) + Z_2 \sin(ct)$ has autocovariance function $h \mapsto \sigma^2 \cos(ch)$. Therefore, the process
\[  A\cos(\pi t / 3) + B\sin(\pi t / 3) \]
is an stationary process with autocovariance function $\gamma_1(h) = v^2 \cos(\pi h/3)$.

On the other hand, according to Example 3.1.1, the autocovariance function for a $\MA(q)$ process $Y_t = \theta(B)Z_t$ is
\[ \gamma_2(h) = \1_{|h| \leq q} \sigma^2 \sum_{j = 0}^{q-|h|}\theta_j \theta_{j+h}. \]
Thus, for the $\MA(1)$ process $Y_t = Z_t + 2.5 Z_{t-1}$,
\[ \gamma_2(0) = \sigma^2 (1+2.5^2) = 7.25\sigma^2,\quad \gamma_2(1) = 2.5 \sigma^2. \]
Finally, in\textbf{ Exercise 1.11} we proved that the autocovariance of the sum of uncorrelated stationary processes is the sum of their respective autocovariance functions. Thus, the autocovariance function for $X_t$ is
\[ \gamma(h) = \gamma_1(h) + \gamma_2(h). \]

\subsection*{Solution Spectral Distribution}

According to equation (4.2.3), the spectral distribution $F_1$ of $\gamma_1$ must satisfy
\[ \gamma_1(h) = \int_{-\pi}^\pi e^{ihv} d F(v).  \]
Let $u = \pi t /3$. Note that we cannot take the spectral density
\[ \everymath{\displaystyle}
\arraycolsep=1.8pt\def\arraystretch{2.5}
\begin{array}{rcl}
    f_1(\lambda) & = & \frac{1}{2\pi} v^2 \sum_{n = -\infty}^{\infty} e^{-in\lambda} \cos(n u)\\
    & = & \frac{v^2}{2\pi} + \frac{v^2}{\pi} \sum_{n = 1}^{\infty} \cos(n\lambda) \cos(n u) = \infty
\end{array}\]
because it diverges when $\lambda = \pm u$. Instead, since $\gamma$ is non-definite, $F$ must exist, so compute the spectral distribution by taking the antiderivative of every term directly:
\[ \everymath{\displaystyle}
\arraycolsep=1.8pt\def\arraystretch{2.5}
\begin{array}{rcl}
    F(\lambda) 
    & = & \frac{v^2}{2\pi} \sum_{\substack{n \in \Z\\ n \neq 0}} \frac{e^{-in\lambda}}{-in} \cos(n u)\\
    & = & \frac{v^2}{2\pi} \sum_{\substack{n \in \Z\\ n \neq 0}} \frac{1}{-2ni} \left( e^{-i(\lambda - u)n} + e^{-i(\lambda + u)n}  \right)\\
    \mbox{\tiny $(y(z) = e^{iz})$}& = & \frac{-v^2}{4\pi i} \sum_{n = 1}^{\infty} \frac{1}{n} (y(-\lambda + u)^n + y(-\lambda - u)^n - y(\lambda - u)^n - y(\lambda + u)^n)
\end{array} \]

Then, using the identity $\sum_{n = 1}^{\infty} y^n/n = -\ln(1-y)$ for $|y| \leq 1$ and $y \neq -1$, we obtain
\[ \everymath{\displaystyle}
\arraycolsep=1.8pt\def\arraystretch{2.5}
\begin{array}{rcl}
    \cdots & = & \frac{-v^2}{4\pi i} (-\ln(1-y(-\lambda + u)) - \ln(1-y(-\lambda - u)) + \ln(1-y(\lambda - u)) + \ln(1-y(\lambda + u)))\\
    & = & \frac{-v^2}{4\pi i} \ln\left( \frac{(1-e^{(-\lambda + u)i})}{(1-e^{(\lambda - u)i})} \cdot \frac{(1-e^{(-\lambda - u)i})}{(1-e^{(\lambda + u)i})} \right)\\
\end{array} \]
Use the identity $(1-x)/(1-x^{-1}) = -x$ with $x = e^{(-\lambda\pm u)}$ to obtain
\[ \everymath{\displaystyle}
\arraycolsep=1.8pt\def\arraystretch{2.5}
\begin{array}{rcl}
    \cdots & = &\frac{-v^2}{4\pi i} (\ln(e^{(-\lambda + u)i}) + \ln(e^{(-\lambda - u)i}))\\
    & = & \frac{-v^2}{4\pi i} (-2\lambda)i = \frac{v^2}{2\pi} \lambda.
\end{array} \]

On the other hand, in \textbf{Example 4.4.1.} we proved that the spectral density for any $\MA(1)$ process $Y_t = Z_t + \theta Z_{t-1}$ is 
\[ f_2(\lambda) = \frac{\sigma^2}{2\pi} (1+2\theta \cos(\lambda) + \theta^2). \]
For our case, $\theta = 2.5$, so 
\[ f_2(\lambda) = \frac{\sigma^2}{2\pi} (5 \cos(\lambda) + 7.25), \]
and thus,
\[ F_2(\lambda) = \frac{\sigma^2}{2\pi} (5 \sin(\lambda) + 7.25 \lambda). \]
Finally, by linearity of the integral we obtain that the spectral density of $X_t$ is
\[ F(\lambda) = F_1(\lambda) + F_2(\lambda) = \frac{1}{2\pi}(5\sigma^2 \sin(\lambda) + (7.25 \sigma^2 + v^2)\lambda). \]
