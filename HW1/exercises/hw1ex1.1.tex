% chktex-file 3 chktex-file 18 chktex-file 36 chktex-file 40
\section*{Exercise 1.1.}

Suppose that $X_t = Z_t + \theta Z_{t-1},\; t = 1,2,\ldots,$ where $Z_0, Z_1, Z_2$ are independent random variables, each with moment generating function $\E \exp(\lambda Z_i) = m(\lambda)$.
\begin{enumerate}[label=(\alph*)]
    \item Express the joint moment generating function $\E\exp\left( \sum_{i = 1}^{n}\lambda_i X_i \right)$ in terms of the function $m(\cdot)$.
    \item Deduce from (a) that $\{X_t\}$ is strictly stationary.
\end{enumerate}

\subsection*{Solution part (a)}

Since $\{Z_t\}$ are independent, for $X_t = Z_t + \theta Z_{t-1}$, the moment generating function:
\[ \everymath{\displaystyle}
\arraycolsep=1.8pt\def\arraystretch{2.5}
\begin{array}{rcl}
    \E \exp(\lambda X_t) & = & \E\exp(\lambda (Z_t + \theta Z_{t-1}))\\
    & = & \E\exp(\lambda Z_t)\cdot \E\exp(\lambda \theta \Z_{t-1}) \\
    & = & m(\lambda)\cdot m(\theta\lambda)
\end{array} \]
On the other hand,
\[ \everymath{\displaystyle}
\arraycolsep=1.8pt\def\arraystretch{2.5}
\begin{array}{rcl}
    \sum_{i = 1}^{n} \lambda_i X_i  & = & \sum_{i = 1}^{n} \lambda_i Z_i + \sum_{i = 1}^{n} \lambda_i \theta Z_{i-1}\\
    & = & \sum_{i = 1}^{n} \lambda_i Z_i + \sum_{i = 0}^{n-1} \lambda_{i+1} \theta Z_{i}\\
    & = & \lambda_n Z_n + \left[ \sum_{i = 1}^{n-1} (\lambda_i + \theta \lambda_{i+1}) Z_i \right] + \lambda_1 \theta Z_0.
\end{array} \]
Therefore, using a similar argument as before
\[ \everymath{\displaystyle}
\arraycolsep=1.8pt\def\arraystretch{2.5}
\begin{array}{rcl}
    \E \exp\left( \sum_{i = 1}^{n} \lambda_i X_i \right) = m(\lambda_n) \cdot \left[ \prod_{i = 1}^{n-1} m(\lambda_i + \theta \lambda_{i+1}) \right] \cdot m(\lambda_1 \theta)
\end{array} \]

\subsection*{Solution part (b)}

Let $(X_{1},\ldots, X_{n})'$ be a random vector in $\R^{k}$. The moment generating function of this vector is defined as follows, 
\[ M_{X_{1:n}}(\lambda_{1:n}) = \E \exp(\angles{\lambda_{1:n},X_{1:n}}) = \E \exp\left( \sum_{i = 1}^{n} \lambda_i X_i \right),\; \lambda_{1:n} \in \R^n. \]
So, we know from the previous part that
\[ \everymath{\displaystyle}
\arraycolsep=1.8pt\def\arraystretch{2.5}
\begin{array}{rcl}
    M_{X_{1:n}}(\lambda_{1:n})& = & m(\lambda_n) \cdot \left[ \prod_{i = 1}^{n-1} m(\lambda_i + \theta \lambda_{i+1}) \right] \cdot m(\lambda_1 \theta)\\
    & = & \E\exp(\lambda_n Z_{n+h}) + \left[ \prod_{i = 1}^{n-1} \E \exp((\lambda_i + \theta \lambda_{i+1}) Z_{i+h}) \right] + \E\exp(\lambda_1 \theta Z_{h})\\
    & = & \E \exp\left( \sum_{i = 1}^{n} \lambda_i X_{i+h} \right)\\
    & = & M_{X_{1+h:n+h}}(\lambda_{1:n})
\end{array} \]
Since the moment generating function of both $(X_{1},\ldots, X_{n})'$ and $(X_{1+h},\ldots, X_{n+h})'$ coincide, they have the same joint distribution. Thus, $\{X_t\}$ is strictly stationary.