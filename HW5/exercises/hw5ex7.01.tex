% chktex-file 3 chktex-file 12 chktex-file 18 chktex-file 36 chktex-file 40
\section*{Exercise 7.1.}

% 5.8 (a) (b) (leer también sección 5.4), 5.11, 7.1, 7.3 y 7.4.

If $\{X_{t} \}$ is a causal \textbf{AR}(1) process with mean $\mu,$ show that ${\bar{X}}_{n}$ is $\mathrm{A N} ( \mu,$  $\sigma^{2} ( 1-\phi)^{-2} n^{-1} )$ . In a sample of size 100 from an \textbf{AR}(1) process with $\phi=. 6$ and $\sigma^{2}=2$, we obtain $\bar{X}_{n}=. 2 7 1$. Construct an approximate 95\% confidence interval for the mean $\mu$. Does the data suggest that $\mu=0$?

\subsection*{Solution}

For the causal \textbf{AR}(1) process with mean $\mu$ in this exercise we have that for some $\{Z_t\} \sim \mbox{IID}(0,\sigma^2)$ (otherwise, if $Z_t$ is not IID, then I don't know how to use Theorem 7.1.2),
\[ Y_t = X_t - \mu, \]
\[ Y_t - \phi Y_{t-1} = Z_t. \]
Then, since $\{X_t\}$ is causal, $|\phi| < 1$, so it follows that
\[ \psi(z) = \frac{\Theta(z)}{\Phi(z)} = \frac{1}{1-\phi z} = \sum_{k = 0}^{\infty} \phi^k z^k, \]
\[ \implies \psi_n = \phi^n \]
and thus,
\[ \everymath{\displaystyle}
\arraycolsep=1.8pt\def\arraystretch{2.5}
\begin{array}{rcl}
    v & = & \sigma^2 \left( \sum_{n = 0} \psi_n \right)^2\\
    & = & \sigma^2 \left( \sum_{n = 0} \phi^n \right)^2\\
    & = & \frac{\sigma^2}{(1-\phi)^2}.
\end{array}  \]
Then, using Theorem 7.1.2, we conclude that
\[ \ol{X}_n = n^{-1}(X_1+\cdots,X_n) \sim \mbox{AN}(\mu, \sigma^2(1-\phi)^2 n^{-1}) \]
In order to find the bounds for $\mu$, note that
\[ n^{1/2}(\overline{{{X}}}_{n}-\mu)\sim W \sim \mathrm{N}\left(0,\sum_{|h|<n}\left(1\,-\frac{|h|}{n}\right)\gamma(h)\right),
\]
In our case, $\sigma^2 = 4$, $\phi = 0.6$, and therefore,
\[ \everymath{\displaystyle}
\arraycolsep=1.8pt\def\arraystretch{2.5}
\begin{array}{rcl}
    \gamma(h) & = & \sum_{n = 0}^{\infty} \psi_n \psi_{n+|k|}\\
    & = & \left( \frac{3}{5} \right)^{|k|}  \sum_{n = 0}^{\infty} \left( \frac{3}{5} \right)^{2n}\\
    & = & \left( \frac{3}{5} \right)^{|k|}\cdot \frac{25}{16}
\end{array}  \]

After putting the numbers in the calculator, we obtain,
\[ \sum_{h = -99}^{99}\left(1\,-\frac{|h|}{100}\right)\gamma(h) = 6.1328125. \]

Finally, the bounds of the $95\%$ confidence interval are
\[ \Phi_{\alpha/2} = -4.85375593,\hspace{2em} \Phi_{1-\alpha/2} = 4.85375593 \]
Therefore, $1-\alpha = 95\%$ confidence interval for $\mu = 100^{-1/2} W - \ol{X}_n = W/10 - 0.271$ is
\[ I_{\alpha}^{(\mu)} = [-0.75637559, 0.21437559], \]

so is somewhat implausible for $\mu$ to be 0 because is beyond a $95\%$ confidence interval.