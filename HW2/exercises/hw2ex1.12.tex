% chktex-file 3 chktex-file 18 chktex-file 36 chktex-file 40
\section*{Exercise 1.12}

Which, if any, of the following functions defined on the integers is the autocovariance function of a stationary time series?

\begin{enumerate}
    \item[(b)] $f(h) = (-1)^{|h|},$
    \item[(d)] $f(h) = 1 + \cos \frac{\pi h}{2} - \cos \frac{\pi h}{4},$
    \item[(f)] $f(h) = \begin{cases}
        1 & \mbox{if } h = 0,\\
        .6 & \mbox{if } h = \pm 1,\\
        0 & \mbox{otherwise}.
    \end{cases} $
\end{enumerate}

\subsection*{Solution Item (b)}

For every $n\in \N$, the matrix
\[ M_n = [f(i-j)]_{i,j = 1}^n = \left[ \begin{matrix}
    1 & -1 & \cdots\\
    -1 & 1 & \cdots\\
    \vdots & \vdots & \ddots
\end{matrix} \right], \] 
has eigenvalue $0$ with multiplicity $n-1$ (the first column is repeated $n$ times) and eigenvalue $n$ with multiplicity $1$:
\[ M \left[ \begin{matrix}
    1\\-1\\\vdots
\end{matrix} \right] = \left[ \begin{matrix}
    n\\-n\\\vdots
\end{matrix} \right] .\]
Therefore, since all eigenvalues are non-negative, the function is positive semi-definite, so it corresponds to an autocovariance function of an stationary process.

\subsection*{Solution Item (d)}

With the following code I found numerically that the matrix $[f(i-j)]_{i,j = 1}^n$ is not semidefinite positive when $n = 4$.

\begin{minted}{python}
    import numpy as np 
    cos = np.cos
    sin = np.sin
    pi = np.pi
    n = 4
    matrix = np.matrix([[
        1 + cos(pi*(i-j)/2) - cos(pi*(i-j)/4) 
        for i in range(1,n+1)] for j in range(1,n+1)])
    print(np.linalg.eigvals(matrix).round(10).astype(float))
\end{minted}

This last line returns \mintinline{python}{array([-0.76536686,  2.76536686,  0.76536686,  1.23463314])}, which has a negative eigenvalue. The numerical error of the algorithm is low enough to consider the negative eigenvalue close enough to the real eigenvalue.

\subsection*{Solution Item (f)}

Let $M_n$ be a $n\times n$ tridiagonal (Toeplitz) matrix with the following form
\[ M_n = \left[ \begin{matrix}
    a & c\\
    b & a & c\\
    &\ddots & \ddots&\ddots\\
    & & b & a & c\\
    & & & b & a
\end{matrix} \right] .\]
The eigenvalues of this matrix are 
\[ \lambda_k = a + 2\sqrt{bc} \cos\frac{k\pi}{n+1},\hspace{1em} k = 1,\ldots, n. \]
For our case, $a = f(0) = 1$, $b = c = f(\pm 1) = 0.6$, 
\[ \lambda_k = 1 + 1.2 \cos \frac{k\pi}{n+1}, \]
so $\lambda_{n} < 1$ for a big enough $n$. In fact, by taking $n = 5$, we have that the smallest eigenvalue is approximately $\lambda_n = -0.0392$ (this was done numerically as the previous exercise)

% (b), (d), (f)

\subsection*{Solution Item (c)}

\textbf{Note:} I solved by mistake this exercise, so I'm leaving the solution here.

The book suggested to find a time series for which $\gamma(h) = 1 + \cos \frac{\pi h}{2} + \cos \frac{\pi h}{4}$. My idea was to use what I proved in exercise 1.7.(c) in the previous homework:
\[ A_t = a Z_1, \]
\[ B_t = b Z_2 \cos \frac{\pi t}{2} + b Z_3 \sin \frac{\pi t}{2},  \]
\[ C_t = c Z_4 \cos \frac{\pi t}{4} + c Z_5 \sin \frac{\pi t}{4}, \]
for independent $Z_t \sim N(0,1)$. Therefore,
\[ \gamma_A(h) = a^2,\hspace{1em}\gamma_B(h) = b^2 \cos \frac{\pi h}{2},\hspace{1em} \gamma_C(h) =  c^2 \cos \frac{\pi h}{4}  \]
Then, since $A_t$, $B_t$ and $C_t$ are uncorrelated, we can use the previous exercise to assert that if
\[ X_t = A_t + B_t + C_t,\]
then the autocovariance function is
\[ \everymath{\displaystyle}
\arraycolsep=1.8pt\def\arraystretch{2}
\begin{array}{rcl}
    \gamma_X(h) & = & \gamma_A(h) + \gamma_B(h) + \gamma_C(h)\\
    & = & 1 + \cos \frac{\pi h}{2} + \cos \frac{\pi h}{4}.
\end{array}\]
By choosing $a = b = c = 1$ we'd prove that $\gamma_X = f$ is a semidefinite positive function.